\documentclass[a4paper]{article}

\usepackage{amsmath}
\usepackage{amssymb}
\usepackage{cmap}
\usepackage{geometry}
\usepackage{hyperref}
\usepackage{indentfirst}
\usepackage{xeCJK}

\geometry{margin=1in}

\setCJKmainfont[BoldFont={SimHei}]
{SimSun}
\setCJKmonofont{FangSong}

\title{2012年汕头市青少年信息学奥林匹克决赛\\
提高组}
%\author{ftiasch}
\date{2012年4月15日}

\renewcommand{\thesection}{第\arabic{section}题}
\renewcommand{\thesubsection}{}

\newcommand{\problem}{\section}
\newcommand{\inputformat}{\subsection{输入格式}}
\newcommand{\outputformat}{\subsection{输出格式}}
\newcommand{\sample}[2]{
\subsection{输入输出样例}
\begin{tabular}{|l|l|}
    \hline
    输入 & 输出 \\
    \hline
    \begin{minipage}[t]{200pt}
        \begin{ttfamily} #1 \end{ttfamily}
    \end{minipage} &
    \begin{minipage}[t]{200pt}
        \begin{ttfamily} #2 \end{ttfamily}
    \end{minipage} \\
    \hline
\end{tabular}
\vspace{1ex}
\par
}
\newcommand{\dataset}{\subsection{数据范围}}

\begin{document}

\maketitle

\begin{enumerate}
\item 参赛选手提交\textbf{源文件},文件名为对应题目英文加上语言后缀。例如,test一题的提交应该名为test.pas, test.c或者test.cpp。\textbf{不要}在提交目录下建立子目录。

\item 程序需要使用\textbf{文件}输入输出,输入输出文件名为对应题目英文加上文件类型。例如,test一题的输入输出文件分别为test.in和test.out。

\item 时间限制原则上至少是标准程序运行时间的$2$倍,空间限制为$256$M。

\item 最终解释权归出题人所有。
\end{enumerate}

\problem{数学(math)}

ftiasch和nm是好朋友。nm的智商比较低,于是ftiasch经常会出一些数学题来提高nm的智商。这次,他又出了一道题目:

给出数列$A_1, A_2, \ldots, A_N$,并设\[B_i = \frac{A_1 \cdot A_2 \cdots A_N}{A_i} \bmod (10^9 + 7)\]

请你帮助nm,把所有的$B_i$算出来。

\inputformat{}

第$1$行,$1$个整数$N$, 表示数列的长度。第$2$行,$N$个整数$A_1, A_2, \ldots, A_N$,表示给出的数列。

\outputformat{}

第$1$行,$N$个整数,表示算出的$B_1, B_2, \ldots, B_N$。

\sample{
3 \\
1 2 3 \\
}{
6 3 2 \\
}

\dataset{}

\begin{itemize}
    \item 对于$50\%$的数据,$1 \leq N \leq 1,000$。

    \item 对于$100\%$的数据,$1 \leq N \leq 100,000$,$1 \leq A_i \leq 10^9$。
\end{itemize}

\problem{黑屋(trap)}

由于nm没有把题目做出来,ftiasch把nm关进了小黑屋。小黑屋是矩形的,被纵向划分成了$N$行、横向划分成了$M$列房间。如果从高处往下看,这小黑屋就是被划分成了$N$行$M$列。记第$i$行第$j$列的房间为$(i, j)$,则左上角的房间为$(1, 1)$,右下角的房间为$(n, m)$。nm可以向上下左右移动,就是说,如果nm在$(x, y)$,他可以移动到$(x - 1, y)$, $(x + 1, y)$, $(x, y - 1)$, $(x, y + 1)$其中任意的位置。注意,由于是小黑屋,nm不能移动出小黑屋的边界。

现在,nm有这样一个问题:他现在位于$(x_0, y_0)$,他想知道,如果移动$K$次,一共有多少合法的方案。

\inputformat{}

第$1$行,$5$个整数$N$, $M$, $K$, $x_0$, $y_0$。

\outputformat{}

第$1$行,$1$个整数,代表合法方案的数量。因为数值太大,你只需要输出结果除以$(10^9 + 7)$的余数。

\sample{
1 2 1 1 2 \\
}{
1 \\
}

\dataset{}

\begin{itemize}
    \item 对于$50\%$的数据,$1 \leq N, M, K \leq 100$。

    \item 对于$100\%$的数据,$1 \leq N, M, K \leq 1,000$,$1 \leq x_0 \leq N$,$1 \leq y_0 \leq M$。
\end{itemize}

\problem{取球(ball)}

被关一轮小黑屋后,nm终于被ftiasch放出来了。他们玩起了这样一个游戏:

$N$个球排成一列,每个球可能是红(a)、绿(b)、蓝(c)三种颜色之一。如果相邻两个球的颜色不一样,则可以拿走这两个球,在这个位置换上一个第三种颜色的球。例如,现在有这样一列球:bbcc,经过一次操作后,则可以变成bac。显然,操作不可能无限进行,而且对于不同的操作方式,剩下球的数量也不相同。

现在,你作为裁判,需要知道在最优策略下,最少能剩下多少个球。

\inputformat{}

第$1$行,$1$个字符串,表示初始时球的颜色。

\outputformat{}

第$1$行,$1$个整数,表示最少剩下的球数。

\sample{
abc \\
}{
2 \\
}

\dataset

\begin{itemize}
    \item 对于$50\%$的数据,$1 \leq N \leq 100$。

    \item 对于$100\%$的数据,$1 \leq N \leq 5,000$,输入的字符串只包含a, b, c三种字符。
\end{itemize}

\problem{异或(xor)}

这天nm又遇到一道奇怪的题目,这道题目是这样的:

给定一个数列$A_1, A_2, \ldots, A_N$,会有很多次询问,每次询问一个区间$[L_i, R_i]$,并给出一个$B_i$,求$A_k \oplus B_i(L_i \leq k \leq R_i)$的最大值是多少,其中$A \oplus B$表示$A$和$B$的异或值。

nm表示不会做,于是去问ftiasch。ftiasch表示,这么水的题目一点意思都没有,然后就扔给你做了。

\inputformat{}

第$1$行,$2$个整数$N$, $M$, 表示数列的长度和询问的数量。

第$2$行,$N$个整数$A_1, A_2, \ldots, A_N$, 表示给出的数列。

第$3 \sim (2 + M)$行,每行$3$个整数$L_i, R_i, B_i$。

\outputformat{}

$M$行,每行$1$个整数,代表每个询问的最大值。

\sample{
3 2 \\
1 3 2 \\
1 2 1 \\
1 3 1 \\
}{
2 \\
3 \\
}

\dataset

\begin{itemize}
    \item 对于$50\%$的数据,$1 \leq N, M \leq 1,000$。

    \item 对于$100\%$的数据,$1 \leq N, M \leq 50,000$,$0 \leq A_i, B_i \leq 10,000$,$1 \leq L_i \leq R_i \leq N$。
\end{itemize}

\end{document}
